% !TeX root=main.tex
% در این فایل، عنوان پایان‌نامه، مشخصات خود، متن تقدیمی‌، ستایش، سپاس‌گزاری و چکیده پایان‌نامه را به فارسی، وارد کنید.
% توجه داشته باشید که جدول حاوی مشخصات پروژه/پایان‌نامه/رساله و همچنین، مشخصات داخل آن، به طور خودکار، درج می‌شود.
%%%%%%%%%%%%%%%%%%%%%%%%%%%%%%%%%%%%
% دانشگاه خود را وارد کنید
\university{علم و صنعت ایران}
% دانشکده، آموزشکده و یا پژوهشکده  خود را وارد کنید
\faculty{دانشکده مهندسی کامپیوتر}
% گروه آموزشی خود را وارد کنید
\department{گروه هوش مصنوعی و رباتیک}
% گروه آموزشی خود را وارد کنید
\subject{مهندسی کامپیوتر}
% گرایش خود را وارد کنید
\field{هوش مصنوعی و رباتیک}
% عنوان پایان‌نامه را وارد کنید
\title{طراحی و پیاده‌سازی شبکه اجتماعی محلی و بی‌سیم بدون اینترنت}
% نام استاد(ان) راهنما را وارد کنید
\firstsupervisor{سید صالح اعتمادی}
%\secondsupervisor{استاد راهنمای دوم}
% نام استاد(دان) مشاور را وارد کنید. چنانچه استاد مشاور ندارید، دستور پایین را غیرفعال کنید.
%\firstadvisor{استاد مشاور اول}
%\secondadvisor{استاد مشاور دوم}
% نام دانشجو را وارد کنید
\name{مریم سادات}
% نام خانوادگی دانشجو را وارد کنید
\surname{هاشمی}
% شماره دانشجویی دانشجو را وارد کنید
\studentID{94523252}
% تاریخ پایان‌نامه را وارد کنید
\thesisdate{آبان ۱۳۹۸}
% به صورت پیش‌فرض برای پایان‌نامه‌های کارشناسی تا دکترا به ترتیب از عبارات «پروژه»، «پایان‌نامه» و »رساله» استفاده می‌شود؛ اگر  نمی‌پسندید هر عنوانی را که مایلید در دستور زیر قرار داده و آنرا از حالت توضیح خارج کنید.
\projectLabel{پروژه}

% به صورت پیش‌فرض برای عناوین مقاطع تحصیلی کارشناسی تا دکترا به ترتیب از عبارات «کارشناسی»، «کارشناسی ارشد» و »دکترا» استفاده می‌شود؛ اگر  نمی‌پسندید هر عنوانی را که مایلید در دستور زیر قرار داده و آنرا از حالت توضیح خارج کنید.
\degree{کارشناسی}

\firstPage
\besmPage

\esalatPage
\mojavezPage


% چنانچه مایل به چاپ صفحات «تقدیم»، «نیایش» و «سپاس‌گزاری» در خروجی نیستید، خط‌های زیر را با گذاشتن ٪  در ابتدای آنها غیرفعال کنید.
 % پایان‌نامه خود را تقدیم کنید!

%\newpage
%\thispagestyle{empty}
%{\Large تقدیم به:}\\
%\begin{flushleft}
%{\huge
%همسر و فرزندانم\\
%\vspace{7mm}
%و\\
%\vspace{7mm}
%پدر و مادرم
%}
%\end{flushleft}


% سپاس‌گزاری
\begin{acknowledgementpage}
سپاس خداوندگار حکیم را که با لطف بی‌کران خود، آدمی را زیور عقل آراست.


در آغاز وظیفه‌  خود  می‌دانم از زحمات بی‌دریغ استاد  راهنمای خود،  جناب آقای دکتر سید صالح اعتمادی، صمیمانه تشکر و  قدردانی کنم  که قطعاً بدون راهنمایی‌های ارزنده‌  ایشان، این مجموعه  به انجام  نمی‌رسید.

همچنین لازم می‌دانم از پدید آورندگان بسته زی‌پرشین، مخصوصاً جناب آقای  وفا خلیقی، که این پایان‌نامه با استفاده از این بسته، آماده شده است و همه دوستانمان در گروه پارسی‌لاتک کمال قدردانی را داشته باشم.

 در پایان، بوسه می‌زنم بر دستان خداوندگاران مهر و مهربانی، پدر و مادر عزیزم و بعد از خدا، ستایش می‌کنم وجود مقدس‌شان را و تشکر می‌کنم از خانواده عزیزم به پاس عاطفه سرشار و گرمای امیدبخش وجودشان، که بهترین پشتیبان من بودند.
% با استفاده از دستور زیر، امضای شما، به طور خودکار، درج می‌شود.
\signature 
\end{acknowledgementpage}
%%%%%%%%%%%%%%%%%%%%%%%%%%%%%%%%%%%%
% کلمات کلیدی پایان‌نامه را وارد کنید
\keywords{شبکه اجتماعی، بی‌سیم، شبکه مش، اپلیکیشن موبایل،  همسایگی}
%چکیده پایان‌نامه را وارد کنید، برای ایجاد پاراگراف جدید از \\ استفاده کنید. اگر خط خالی دشته باشید، خطا خواهید گرفت.
\fa-abstract{
هدف این پروژه اضافه کردن لایه خدمات شبکه اجتماعی به یک شبکه ارتباطی بی‌سیم مش می‌باشد. در این پروژه ابتدا یک اپلیکیشن موبایل برای ارائه سرویس همسایگی پیاده‌سازی می‌شود. سپس خدمات لازم از لایه خدمات اجتماعی برای این سرویس شناسایی می‌شوند. سپس خدمات این لایه برای ارائه روی شبکه ارتباطی بی‌سیم مش باز‌مهندسی، طراحی و پیاده‌سازی می‌شوند. در نهایت با استفاده از این اپلیکیشن می‌توانیم شبکه‌ی اجتماعی را راه‌اندازی کنیم که حتی بدون اتصال به اینترنت قادر به ارائه خدمات اجتماعی برخط مثل ارسال پیغام‌های شخصی یا عمومی می‌باشد.
}

\abstractPage

\newpage\clearpage