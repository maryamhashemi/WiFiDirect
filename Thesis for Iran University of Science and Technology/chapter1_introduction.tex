% !TeX root=main.tex
% دستور زیر باید در اولین فصل شما باشد. آن را حذف نکنید!
\pagenumbering{arabic}

\chapter{مقدمه}
\thispagestyle{empty}
یکی از نیازهای اساسی انسان‌ها نیاز به برقراری ارتباط با دیگران و اجتماعی شدن است. پایه ای‌ترین ارتباطات اجتماعی را می‌توان خانواده، بستگان، دوستان و همکاران دانست که روابط معناداری بین افراد در این گروه‌ها برقرار است. این در حالی است که در سال‌های اخیر فناوری دیجیتال و اینترنت شکل جدیدی از ارتباط را تعریف کرده‌اند و در حال حاضر شاهد گسترش بی سابقه شبکه‌های اجتماعی
\LTRfootnote{Social Networks}
 در بین مردم هستیم. به گونه‌ای که اکنون شبکه‌های اجتماعی جزئی از زندگی روزمره مردم شده‌است. 
 
 شبکه‌های اجتماعی به کاربرانش این فرصت را می‌دهد تا آراء و نظرات خود را با دیگران مطرح کنند و از نظرات مختلف مطلع شوند. در سایت‌های خرید وارد شوند و کالای موردنظر خود را جستجو کنند و نظرات خریداران قبلی را بخوانند و سفارش خود را ثبت کنند. به‌راحتی اطلاعات مقصد و هر آنچه برای یک سفر لازم است را از اینترنت دریافت کرده و از نظرات و راهنمایی‌های افرادی که قبلاً این سفر را تجربه کرده‌اند استفاده کنند و با طیب خاطر به سفر بروند. در فعالیت‌های گروهی و تشکل‌های مردم‌نهاد مشارکت کنند. بسیاری از فعالیت‌های این چنینی که ذاتی اجتماعی دارند با کمک اینترنت و ظهور شبکه‌های اجتماعی برای کاربران آسان شده‌اند.
 
در بین همه‌ی این قابلیت‌ها که شبکه‌های اجتماعی کنونی در اختیار ما قرار می‌دهند، شرایطی وجود دارد که این شبکه‌های اجتماعی پاسخگوی نیاز‌های ما نیستند. مثلا بخواهید خبر گم شدن حیوان خانگیتان را در محله‌ی خود اعلام کنید و یا بخواهید یک نردبان قرض بگیرید و یا به دنبال یک دندان‌پزشکی یا لوله‌کش خوب در نزدیکی محل زندگیتان باشید. تمامی این مثال‌ها نیازمند این است که شما با همسایگانتان و افرادی که در نزدیکی شما و محله تان زندگی می کنند؛ در ارتباط باشید. موضوعی که امروزه در زندگی ما کمرنگ شده‌است. از طرفی روند شبکه‌های اجتماعی مثل فیسبوک
\LTRfootnote{Facebook}
، توییتر 
\LTRfootnote{Twitter}
و ... به گونه‌ای است که ما را با دوستانی که 20 سال گذشته با آن‌ها ارتباط داشتیم، متصل می‌کند اما با افرادی که در نزدیکی ما زندگی می‌کنند و نیاز بیشتری به ارتباط با آن‌ها داریم، متصل نمی‌کند.

بنابراین در این پروژه ما قصد داریم که یک شبکه‌ی اجتماعی محلی را پیاده‌سازی کنیم که امکان برقراری ارتباط با همسایگانمان را برای ما ایجاد‌کند. از طرفی چون هدف ما در این شبکه‌ی اجتماعی اتصال به افرادی است که از نظر بعد جغرافیایی به ما نزدیک هستند، می‌توانیم با استفاده از تکنولوژی‌هایی همچون بلوتوث
\LTRfootnote{Bluetooth}
، اتصال نقطه به نقطه وای فای 
\LTRfootnote{Peer to Peer WiFi Connnection}
و ... این ارتباط را حاصل کنیم بدون این که بخواهیم از شبکه‌ی جهانی اینترنت
\LTRfootnote{Internet}
 استفاده کنیم. برای رسیدن به این منظور از شبکه‌ی بی‌سیم مش 
 \LTRfootnote{Wireless Mesh Network}
 نیز استفاده خواهیم کرد. البته لازم به ذکر است که در حالت ایده آل اگر تعداد کاربران این شبکه‌ی اجتماعی مقدار قابل توجه‌ای در سراسر جهان باشد، می‌توان با کاربران دور دست نیز ارتباط برقرار کرد. 

سایر کاربرد‌های این شبکه‌ی اجتماعی به صورت زیر خواهد بود:

\begin{enumerate}
	\item 
	فرض کنید که شما در نزدیکی دانشکده‌ی مهندسی کامپیوتر قرار دارید و یک رویداد در دانشکده‌ی مهندسی کامپیوتر در حال برگزاری است. بنابراین اپلیکیشن به صورت یک اعلان بر روی گوشی شما، برگزاری رویداد را به شما اطلاع رسانی می‌کند.
	\item 
	فرض کنید شما برای روز چهارشنبه، نیاز به کمک کسی دارید که از فرزندتان نگهداری کند. شما می‌توانید درخواست یک پرستار بچه را در این شبکه محلی به اشتراک بگذارید.
	
	\item 
	شما دنبال یک کارواش، مکانیکی یا دندان پزشکی خوب در نزدیکی خانه‌تان هستید. می‌توانید در مورد این موضوعات از دیگران در این شبکه‌ی محلی بپرسید.
	\item 
	فرض کنید که شما دوچرخه‌تان را اطراف خانه تان گم کرده‌اید. می‌توانید این موضوع را در شبکه محلی اعلان کنید تا اگر پیدا شد دیگران به شما در این شبکه‌ی محلی خبر بدهند یا مثلا می‌توانید این موضوع را به عنوان جرم در یک منطقه یا امن نبودن یک محله اعلام کنید.
	
	\item 
	نهاد‌های دولتی می‌توانند در این شبکه عضو بشوند و ساکنین یک محله را از اتفاقاتی مثل آتش‌سوزی، دزدی، خرابی تلفن، قطع آب و گاز و برق و ... مطلع کنند و یا برعکس به این شکل که در صورت اتفاق افتادن هر یک از این حوادث ساکنین آن محله بتوانند در سریع‌ترین زمان ممکن نهاد دولتی مربوطه را مطلع سازند.
	
	\item 
	فرض کنید که یک سازمان مثل هواشناسی در این شبکه محلی عضو شود و مردم یک شهر، روستا یا یک استان را از آب و هوای آن روز با خبر کند. مثلا امروز در مناطق جنوبی شهر آب گرفتگی داریم یا امروز هوا در ساعات  2 تا 3 بعد از ظهر بارانی است؛ لطفا از چتر استفاده کنید.
	
	\item
	شما می‌توانید زمان رفتن به سر کار خود را به بقیه اعلام کنید و اگر کسی از همسایگانتان در مسیر شما قرار داشت، با شما همراه شود.(کاهش ترافیک و آلودگی هوا)
	\item 
	فروشگاه‌ها و هایپرمارکت‌های محله می‌توانند موجودی کالا‌های خود را اعلام کنند و یا تخفیف‌های اقلام مختلف را در این شبکه‌ی محلی قرار بدهند.
	\item 
	می‌توانید از همسایگانتان در خواست اجاره‌ی حیاط، پارکینگ یا خانه شان را برای برگزاری جلسه‌هاو مراسم‌هایتان و ... داشته باشید.
	\item 
	فرض کنید شما نیاز دارید که بسته‌ای را در نزدیکی محدوده‌ی زندگی‌تان ارسال کنید ولی وقت کافی ندارید و می‌توانید در شبکه درخواست کنید که یک نفر به صورت رایگان این کار را برای شما انجام دهد.
	\item 
	می‌توانید در مورد مسائل مختلف محله‌تان رای گیری کنید مثل تغییر نام یک کوچه.
	 
\end{enumerate}
 
\section{کار‌های مربوطه}
\subsection{\lr{Open Garden}}
 \href{https://www.opengarden.com}{\lr{Open Garden}}%
\LTRfootnote{https://www.opengarden.com}
یک سرویس است که این امکان را به مردم می‌دهد که  خدمات اینترنت خود را با افرادی که در نزدیکی آن‌ها هستند؛ به اشتراک بگذارند. بنابراین با استفاده از این دستگاه اشتراک‌گذاری اینترنت، هر کس می‌تواند پهنای باند اضافی اینترنت خود را ارائه دهد و برای آن پول بگیرد یا خدمات اینترنتی را از دیگران خریداری کند.

\subsection{\lr{NextDoor}} 
 \href{https://nextdoor.com}{\lr{NextDoor}}%
\LTRfootnote{https://nextdoor.com/}
یک سرویس شبکه اجتماعی خصوصی برای محله است. این شرکت در سال 2008 در سانفرانسیسکو، کالیفرنیا تاسیس شد و در اکتبر 2011 در ایالات متحده راه اندازی شد. کاربران 
\lr{NextDoor}
نام و نشانی واقعی خود را به وب سایت ارائه می‌دهند. پیام‌های منتشر شده در وب سایت، فقط برای سایر اعضای  
\lr{NextDoor}
در همان محله قابل مشاهده است.
\subsection{\lr{Fire Chat}}
 \href{https://www.opengarden.com/firechat/}{\lr{FireChat}}%
\LTRfootnote{https://www.opengarden.com/firechat/} 
یک برنامه اختصاصی تلفن همراه است که توسط
\lr{Open Garden} 
توسعه داده شده‌است. این شبکه‌ی اجتماعی از شبکه بی‌سیم مش استفاده می‌کند تا تلفن‌های هوشمند بتوانند از طریق بلوتوث، وای فای یا در چارچوب
\lr{Multipeer} 
شرکت اپل بدون اتصال به اینترنت به یکدیگر متصل شوند. همچنین این برنامه این قابلیت را دارد که از طریق اینترنت نیز اتصال برقرار شود.

\lr{FireChat}
به عنوان یک ابزار ارتباطی در برخی اعتراضات مدنی مورد استفاده قرار گرفت؛ اگر چه که برای چنین اهدافی طراحی نشده‌است. از جمله دلایلی که این شبکه‌ی اجتماعی محبوبیت چندانی بین مردم ندارد؛ می‌توان به ضعف در پیاده‌سازی، عدم اتصال مناسب بین دو دستگاه اندورید
\LTRfootnote{Android}
و آیفون
\LTRfootnote{iphone}
، اشکال در عدم ارسال مناسب عکس و غیره اشاره نمود.    
  
\section{جمع‌بندی}
